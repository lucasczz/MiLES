% This file was adapted from ICLR2022_conference.tex example provided for the ICLR conference
\documentclass{article} % For LaTeX2e
\usepackage{collas2025_conference,times}
\usepackage{easyReview}
\usepackage{booktabs, multicol, multirow}
\usepackage{colortbl}
\usepackage{tikz}
\usepackage{tikz-3dplot}
\usetikzlibrary{arrows.meta} 
\usetikzlibrary{calc}
\usetikzlibrary{positioning}

\usepackage{amsthm}
\usepackage{amsmath}

\DeclareMathOperator*{\concat}{%
    \mathchoice%
        {\Big\Vert}%
        {\big\Vert}%
        {\Vert}%
        {\Vert}%
}

\theoremstyle{definition}
\newtheorem{definition}{Definition}[section]

% Optional math commands from https://github.com/goodfeli/dlbook_notation.
%%%%% NEW MATH DEFINITIONS %%%%%

\usepackage{amsmath,amsfonts,bm}

% Mark sections of captions for referring to divisions of figures
\newcommand{\figleft}{{\em (Left)}}
\newcommand{\figcenter}{{\em (Center)}}
\newcommand{\figright}{{\em (Right)}}
\newcommand{\figtop}{{\em (Top)}}
\newcommand{\figbottom}{{\em (Bottom)}}
\newcommand{\captiona}{{\em (a)}}
\newcommand{\captionb}{{\em (b)}}
\newcommand{\captionc}{{\em (c)}}
\newcommand{\captiond}{{\em (d)}}

% Highlight a newly defined term
\newcommand{\newterm}[1]{{\bf #1}}


% Figure reference, lower-case.
\def\figref#1{figure~\ref{#1}}
% Figure reference, capital. For start of sentence
\def\Figref#1{Figure~\ref{#1}}
\def\twofigref#1#2{figures \ref{#1} and \ref{#2}}
\def\quadfigref#1#2#3#4{figures \ref{#1}, \ref{#2}, \ref{#3} and \ref{#4}}
% Section reference, lower-case.
\def\secref#1{section~\ref{#1}}
% Section reference, capital.
\def\Secref#1{Section~\ref{#1}}
% Reference to two sections.
\def\twosecrefs#1#2{sections \ref{#1} and \ref{#2}}
% Reference to three sections.
\def\secrefs#1#2#3{sections \ref{#1}, \ref{#2} and \ref{#3}}
% Reference to an equation, lower-case.
\def\eqref#1{equation~\ref{#1}}
% Reference to an equation, upper case
\def\Eqref#1{Equation~\ref{#1}}
% A raw reference to an equation---avoid using if possible
\def\plaineqref#1{\ref{#1}}
% Reference to a chapter, lower-case.
\def\chapref#1{chapter~\ref{#1}}
% Reference to an equation, upper case.
\def\Chapref#1{Chapter~\ref{#1}}
% Reference to a range of chapters
\def\rangechapref#1#2{chapters\ref{#1}--\ref{#2}}
% Reference to an algorithm, lower-case.
\def\algref#1{algorithm~\ref{#1}}
% Reference to an algorithm, upper case.
\def\Algref#1{Algorithm~\ref{#1}}
\def\twoalgref#1#2{algorithms \ref{#1} and \ref{#2}}
\def\Twoalgref#1#2{Algorithms \ref{#1} and \ref{#2}}
% Reference to a part, lower case
\def\partref#1{part~\ref{#1}}
% Reference to a part, upper case
\def\Partref#1{Part~\ref{#1}}
\def\twopartref#1#2{parts \ref{#1} and \ref{#2}}

\def\ceil#1{\lceil #1 \rceil}
\def\floor#1{\lfloor #1 \rfloor}
\def\1{\bm{1}}
\newcommand{\train}{\mathcal{D}}
\newcommand{\valid}{\mathcal{D_{\mathrm{valid}}}}
\newcommand{\test}{\mathcal{D_{\mathrm{test}}}}

\def\eps{{\epsilon}}


% Random variables
\def\reta{{\textnormal{$\eta$}}}
\def\ra{{\textnormal{a}}}
\def\rb{{\textnormal{b}}}
\def\rc{{\textnormal{c}}}
\def\rd{{\textnormal{d}}}
\def\re{{\textnormal{e}}}
\def\rf{{\textnormal{f}}}
\def\rg{{\textnormal{g}}}
\def\rh{{\textnormal{h}}}
\def\ri{{\textnormal{i}}}
\def\rj{{\textnormal{j}}}
\def\rk{{\textnormal{k}}}
\def\rl{{\textnormal{l}}}
% rm is already a command, just don't name any random variables m
\def\rn{{\textnormal{n}}}
\def\ro{{\textnormal{o}}}
\def\rp{{\textnormal{p}}}
\def\rq{{\textnormal{q}}}
\def\rr{{\textnormal{r}}}
\def\rs{{\textnormal{s}}}
\def\rt{{\textnormal{t}}}
\def\ru{{\textnormal{u}}}
\def\rv{{\textnormal{v}}}
\def\rw{{\textnormal{w}}}
\def\rx{{\textnormal{x}}}
\def\ry{{\textnormal{y}}}
\def\rz{{\textnormal{z}}}

% Random vectors
\def\rvepsilon{{\mathbf{\epsilon}}}
\def\rvtheta{{\mathbf{\theta}}}
\def\rva{{\mathbf{a}}}
\def\rvb{{\mathbf{b}}}
\def\rvc{{\mathbf{c}}}
\def\rvd{{\mathbf{d}}}
\def\rve{{\mathbf{e}}}
\def\rvf{{\mathbf{f}}}
\def\rvg{{\mathbf{g}}}
\def\rvh{{\mathbf{h}}}
\def\rvu{{\mathbf{i}}}
\def\rvj{{\mathbf{j}}}
\def\rvk{{\mathbf{k}}}
\def\rvl{{\mathbf{l}}}
\def\rvm{{\mathbf{m}}}
\def\rvn{{\mathbf{n}}}
\def\rvo{{\mathbf{o}}}
\def\rvp{{\mathbf{p}}}
\def\rvq{{\mathbf{q}}}
\def\rvr{{\mathbf{r}}}
\def\rvs{{\mathbf{s}}}
\def\rvt{{\mathbf{t}}}
\def\rvu{{\mathbf{u}}}
\def\rvv{{\mathbf{v}}}
\def\rvw{{\mathbf{w}}}
\def\rvx{{\mathbf{x}}}
\def\rvy{{\mathbf{y}}}
\def\rvz{{\mathbf{z}}}

% Elements of random vectors
\def\erva{{\textnormal{a}}}
\def\ervb{{\textnormal{b}}}
\def\ervc{{\textnormal{c}}}
\def\ervd{{\textnormal{d}}}
\def\erve{{\textnormal{e}}}
\def\ervf{{\textnormal{f}}}
\def\ervg{{\textnormal{g}}}
\def\ervh{{\textnormal{h}}}
\def\ervi{{\textnormal{i}}}
\def\ervj{{\textnormal{j}}}
\def\ervk{{\textnormal{k}}}
\def\ervl{{\textnormal{l}}}
\def\ervm{{\textnormal{m}}}
\def\ervn{{\textnormal{n}}}
\def\ervo{{\textnormal{o}}}
\def\ervp{{\textnormal{p}}}
\def\ervq{{\textnormal{q}}}
\def\ervr{{\textnormal{r}}}
\def\ervs{{\textnormal{s}}}
\def\ervt{{\textnormal{t}}}
\def\ervu{{\textnormal{u}}}
\def\ervv{{\textnormal{v}}}
\def\ervw{{\textnormal{w}}}
\def\ervx{{\textnormal{x}}}
\def\ervy{{\textnormal{y}}}
\def\ervz{{\textnormal{z}}}

% Random matrices
\def\rmA{{\mathbf{A}}}
\def\rmB{{\mathbf{B}}}
\def\rmC{{\mathbf{C}}}
\def\rmD{{\mathbf{D}}}
\def\rmE{{\mathbf{E}}}
\def\rmF{{\mathbf{F}}}
\def\rmG{{\mathbf{G}}}
\def\rmH{{\mathbf{H}}}
\def\rmI{{\mathbf{I}}}
\def\rmJ{{\mathbf{J}}}
\def\rmK{{\mathbf{K}}}
\def\rmL{{\mathbf{L}}}
\def\rmM{{\mathbf{M}}}
\def\rmN{{\mathbf{N}}}
\def\rmO{{\mathbf{O}}}
\def\rmP{{\mathbf{P}}}
\def\rmQ{{\mathbf{Q}}}
\def\rmR{{\mathbf{R}}}
\def\rmS{{\mathbf{S}}}
\def\rmT{{\mathbf{T}}}
\def\rmU{{\mathbf{U}}}
\def\rmV{{\mathbf{V}}}
\def\rmW{{\mathbf{W}}}
\def\rmX{{\mathbf{X}}}
\def\rmY{{\mathbf{Y}}}
\def\rmZ{{\mathbf{Z}}}

% Elements of random matrices
\def\ermA{{\textnormal{A}}}
\def\ermB{{\textnormal{B}}}
\def\ermC{{\textnormal{C}}}
\def\ermD{{\textnormal{D}}}
\def\ermE{{\textnormal{E}}}
\def\ermF{{\textnormal{F}}}
\def\ermG{{\textnormal{G}}}
\def\ermH{{\textnormal{H}}}
\def\ermI{{\textnormal{I}}}
\def\ermJ{{\textnormal{J}}}
\def\ermK{{\textnormal{K}}}
\def\ermL{{\textnormal{L}}}
\def\ermM{{\textnormal{M}}}
\def\ermN{{\textnormal{N}}}
\def\ermO{{\textnormal{O}}}
\def\ermP{{\textnormal{P}}}
\def\ermQ{{\textnormal{Q}}}
\def\ermR{{\textnormal{R}}}
\def\ermS{{\textnormal{S}}}
\def\ermT{{\textnormal{T}}}
\def\ermU{{\textnormal{U}}}
\def\ermV{{\textnormal{V}}}
\def\ermW{{\textnormal{W}}}
\def\ermX{{\textnormal{X}}}
\def\ermY{{\textnormal{Y}}}
\def\ermZ{{\textnormal{Z}}}

% Vectors
\def\vzero{{\bm{0}}}
\def\vone{{\bm{1}}}
\def\vmu{{\bm{\mu}}}
\def\vtheta{{\bm{\theta}}}
\def\va{{\bm{a}}}
\def\vb{{\bm{b}}}
\def\vc{{\bm{c}}}
\def\vd{{\bm{d}}}
\def\ve{{\bm{e}}}
\def\vf{{\bm{f}}}
\def\vg{{\bm{g}}}
\def\vh{{\bm{h}}}
\def\vi{{\bm{i}}}
\def\vj{{\bm{j}}}
\def\vk{{\bm{k}}}
\def\vl{{\bm{l}}}
\def\vm{{\bm{m}}}
\def\vn{{\bm{n}}}
\def\vo{{\bm{o}}}
\def\vp{{\bm{p}}}
\def\vq{{\bm{q}}}
\def\vr{{\bm{r}}}
\def\vs{{\bm{s}}}
\def\vt{{\bm{t}}}
\def\vu{{\bm{u}}}
\def\vv{{\bm{v}}}
\def\vw{{\bm{w}}}
\def\vx{{\bm{x}}}
\def\vy{{\bm{y}}}
\def\vz{{\bm{z}}}

% Elements of vectors
\def\evalpha{{\alpha}}
\def\evbeta{{\beta}}
\def\evepsilon{{\epsilon}}
\def\evlambda{{\lambda}}
\def\evomega{{\omega}}
\def\evmu{{\mu}}
\def\evpsi{{\psi}}
\def\evsigma{{\sigma}}
\def\evtheta{{\theta}}
\def\eva{{a}}
\def\evb{{b}}
\def\evc{{c}}
\def\evd{{d}}
\def\eve{{e}}
\def\evf{{f}}
\def\evg{{g}}
\def\evh{{h}}
\def\evi{{i}}
\def\evj{{j}}
\def\evk{{k}}
\def\evl{{l}}
\def\evm{{m}}
\def\evn{{n}}
\def\evo{{o}}
\def\evp{{p}}
\def\evq{{q}}
\def\evr{{r}}
\def\evs{{s}}
\def\evt{{t}}
\def\evu{{u}}
\def\evv{{v}}
\def\evw{{w}}
\def\evx{{x}}
\def\evy{{y}}
\def\evz{{z}}

% Matrix
\def\mA{{\bm{A}}}
\def\mB{{\bm{B}}}
\def\mC{{\bm{C}}}
\def\mD{{\bm{D}}}
\def\mE{{\bm{E}}}
\def\mF{{\bm{F}}}
\def\mG{{\bm{G}}}
\def\mH{{\bm{H}}}
\def\mI{{\bm{I}}}
\def\mJ{{\bm{J}}}
\def\mK{{\bm{K}}}
\def\mL{{\bm{L}}}
\def\mM{{\bm{M}}}
\def\mN{{\bm{N}}}
\def\mO{{\bm{O}}}
\def\mP{{\bm{P}}}
\def\mQ{{\bm{Q}}}
\def\mR{{\bm{R}}}
\def\mS{{\bm{S}}}
\def\mT{{\bm{T}}}
\def\mU{{\bm{U}}}
\def\mV{{\bm{V}}}
\def\mW{{\bm{W}}}
\def\mX{{\bm{X}}}
\def\mY{{\bm{Y}}}
\def\mZ{{\bm{Z}}}
\def\mBeta{{\bm{\beta}}}
\def\mPhi{{\bm{\Phi}}}
\def\mLambda{{\bm{\Lambda}}}
\def\mSigma{{\bm{\Sigma}}}

% Tensor
\DeclareMathAlphabet{\mathsfit}{\encodingdefault}{\sfdefault}{m}{sl}
\SetMathAlphabet{\mathsfit}{bold}{\encodingdefault}{\sfdefault}{bx}{n}
\newcommand{\tens}[1]{\bm{\mathsfit{#1}}}
\def\tA{{\tens{A}}}
\def\tB{{\tens{B}}}
\def\tC{{\tens{C}}}
\def\tD{{\tens{D}}}
\def\tE{{\tens{E}}}
\def\tF{{\tens{F}}}
\def\tG{{\tens{G}}}
\def\tH{{\tens{H}}}
\def\tI{{\tens{I}}}
\def\tJ{{\tens{J}}}
\def\tK{{\tens{K}}}
\def\tL{{\tens{L}}}
\def\tM{{\tens{M}}}
\def\tN{{\tens{N}}}
\def\tO{{\tens{O}}}
\def\tP{{\tens{P}}}
\def\tQ{{\tens{Q}}}
\def\tR{{\tens{R}}}
\def\tS{{\tens{S}}}
\def\tT{{\tens{T}}}
\def\tU{{\tens{U}}}
\def\tV{{\tens{V}}}
\def\tW{{\tens{W}}}
\def\tX{{\tens{X}}}
\def\tY{{\tens{Y}}}
\def\tZ{{\tens{Z}}}


% Graph
\def\gA{{\mathcal{A}}}
\def\gB{{\mathcal{B}}}
\def\gC{{\mathcal{C}}}
\def\gD{{\mathcal{D}}}
\def\gE{{\mathcal{E}}}
\def\gF{{\mathcal{F}}}
\def\gG{{\mathcal{G}}}
\def\gH{{\mathcal{H}}}
\def\gI{{\mathcal{I}}}
\def\gJ{{\mathcal{J}}}
\def\gK{{\mathcal{K}}}
\def\gL{{\mathcal{L}}}
\def\gM{{\mathcal{M}}}
\def\gN{{\mathcal{N}}}
\def\gO{{\mathcal{O}}}
\def\gP{{\mathcal{P}}}
\def\gQ{{\mathcal{Q}}}
\def\gR{{\mathcal{R}}}
\def\gS{{\mathcal{S}}}
\def\gT{{\mathcal{T}}}
\def\gU{{\mathcal{U}}}
\def\gV{{\mathcal{V}}}
\def\gW{{\mathcal{W}}}
\def\gX{{\mathcal{X}}}
\def\gY{{\mathcal{Y}}}
\def\gZ{{\mathcal{Z}}}

% Sets
\def\sA{{\mathbb{A}}}
\def\sB{{\mathbb{B}}}
\def\sC{{\mathbb{C}}}
\def\sD{{\mathbb{D}}}
% Don't use a set called E, because this would be the same as our symbol
% for expectation.
\def\sF{{\mathbb{F}}}
\def\sG{{\mathbb{G}}}
\def\sH{{\mathbb{H}}}
\def\sI{{\mathbb{I}}}
\def\sJ{{\mathbb{J}}}
\def\sK{{\mathbb{K}}}
\def\sL{{\mathbb{L}}}
\def\sM{{\mathbb{M}}}
\def\sN{{\mathbb{N}}}
\def\sO{{\mathbb{O}}}
\def\sP{{\mathbb{P}}}
\def\sQ{{\mathbb{Q}}}
\def\sR{{\mathbb{R}}}
\def\sS{{\mathbb{S}}}
\def\sT{{\mathbb{T}}}
\def\sU{{\mathbb{U}}}
\def\sV{{\mathbb{V}}}
\def\sW{{\mathbb{W}}}
\def\sX{{\mathbb{X}}}
\def\sY{{\mathbb{Y}}}
\def\sZ{{\mathbb{Z}}}

% Entries of a matrix
\def\emLambda{{\Lambda}}
\def\emA{{A}}
\def\emB{{B}}
\def\emC{{C}}
\def\emD{{D}}
\def\emE{{E}}
\def\emF{{F}}
\def\emG{{G}}
\def\emH{{H}}
\def\emI{{I}}
\def\emJ{{J}}
\def\emK{{K}}
\def\emL{{L}}
\def\emM{{M}}
\def\emN{{N}}
\def\emO{{O}}
\def\emP{{P}}
\def\emQ{{Q}}
\def\emR{{R}}
\def\emS{{S}}
\def\emT{{T}}
\def\emU{{U}}
\def\emV{{V}}
\def\emW{{W}}
\def\emX{{X}}
\def\emY{{Y}}
\def\emZ{{Z}}
\def\emSigma{{\Sigma}}

% entries of a tensor
% Same font as tensor, without \bm wrapper
\newcommand{\etens}[1]{\mathsfit{#1}}
\def\etLambda{{\etens{\Lambda}}}
\def\etA{{\etens{A}}}
\def\etB{{\etens{B}}}
\def\etC{{\etens{C}}}
\def\etD{{\etens{D}}}
\def\etE{{\etens{E}}}
\def\etF{{\etens{F}}}
\def\etG{{\etens{G}}}
\def\etH{{\etens{H}}}
\def\etI{{\etens{I}}}
\def\etJ{{\etens{J}}}
\def\etK{{\etens{K}}}
\def\etL{{\etens{L}}}
\def\etM{{\etens{M}}}
\def\etN{{\etens{N}}}
\def\etO{{\etens{O}}}
\def\etP{{\etens{P}}}
\def\etQ{{\etens{Q}}}
\def\etR{{\etens{R}}}
\def\etS{{\etens{S}}}
\def\etT{{\etens{T}}}
\def\etU{{\etens{U}}}
\def\etV{{\etens{V}}}
\def\etW{{\etens{W}}}
\def\etX{{\etens{X}}}
\def\etY{{\etens{Y}}}
\def\etZ{{\etens{Z}}}

% The true underlying data generating distribution
\newcommand{\pdata}{p_{\rm{data}}}
% The empirical distribution defined by the training set
\newcommand{\ptrain}{\hat{p}_{\rm{data}}}
\newcommand{\Ptrain}{\hat{P}_{\rm{data}}}
% The model distribution
\newcommand{\pmodel}{p_{\rm{model}}}
\newcommand{\Pmodel}{P_{\rm{model}}}
\newcommand{\ptildemodel}{\tilde{p}_{\rm{model}}}
% Stochastic autoencoder distributions
\newcommand{\pencode}{p_{\rm{encoder}}}
\newcommand{\pdecode}{p_{\rm{decoder}}}
\newcommand{\precons}{p_{\rm{reconstruct}}}

\newcommand{\laplace}{\mathrm{Laplace}} % Laplace distribution

\newcommand{\E}{\mathbb{E}}
\newcommand{\Ls}{\mathcal{L}}
\newcommand{\R}{\mathbb{R}}
\newcommand{\emp}{\tilde{p}}
\newcommand{\lr}{\alpha}
\newcommand{\reg}{\lambda}
\newcommand{\rect}{\mathrm{rectifier}}
\newcommand{\softmax}{\mathrm{softmax}}
\newcommand{\sigmoid}{\sigma}
\newcommand{\softplus}{\zeta}
\newcommand{\KL}{D_{\mathrm{KL}}}
\newcommand{\Var}{\mathrm{Var}}
\newcommand{\standarderror}{\mathrm{SE}}
\newcommand{\Cov}{\mathrm{Cov}}
% Wolfram Mathworld says $L^2$ is for function spaces and $\ell^2$ is for vectors
% But then they seem to use $L^2$ for vectors throughout the site, and so does
% wikipedia.
\newcommand{\normlzero}{L^0}
\newcommand{\normlone}{L^1}
\newcommand{\normltwo}{L^2}
\newcommand{\normlp}{L^p}
\newcommand{\normmax}{L^\infty}

\newcommand{\parents}{Pa} % See usage in notation.tex. Chosen to match Daphne's book.

\DeclareMathOperator*{\argmax}{arg\,max}
\DeclareMathOperator*{\argmin}{arg\,min}

\DeclareMathOperator{\sign}{sign}
\DeclareMathOperator{\Tr}{Tr}
\let\ab\allowbreak


% Please leave these options as they are
\usepackage{hyperref}
\hypersetup{
    colorlinks=true,
    linkcolor=red,
    filecolor=magenta,
    urlcolor=blue,
    citecolor=purple,
    pdftitle={Overleaf Example},
    pdfpagemode=FullScreen,
    }




\title{Efficient Online Trajectory User Linking with Multi-Level Spatial Embedding Sharing}

% Authors must not appear in the submitted version. They should be hidden
% as long as the \collasfinalcopy macro remains commented out below.
% Non-anonymous submissions will be rejected without review.

\author{Antiquus S.~Hippocampus, Natalia Cerebro  \thanks{ Use footnote for providing further information
about author (webpage, alternative address)---\emph{not} for acknowledging
funding agencies.  Funding acknowledgements go at the end of the paper.} \\
Department of Computer Science\\
Random University\\
Country \\
\texttt{\{hippo,brain\}@cs.random.edu} \\
\And % Use And to have authors side by side
Koala Learnus \& D. Q. ResNet  \\
Department of Computational Neuroscience \\
University of Random City \\
Another Country \\
\texttt{\{koala,net\}@random.rand} \\
\AND % Use AND to have authors block one under the other
Coauthor \\
Affiliation \\
Address \\
\texttt{email}
}

% The \author macro works with any number of authors. There are two commands
% used to separate the names and addresses of multiple authors: \And and \AND.
%
% Using \And between authors leaves it to \LaTeX{} to determine where to break
% the lines. Using \AND forces a linebreak at that point. So, if \LaTeX{}
% puts 3 of 4 authors names on the first line, and the last on the second
% line, try using \AND instead of \And before the third author name.

\newcommand{\fix}{\marginpar{FIX}}
\newcommand{\new}{\marginpar{NEW}}

%\collasfinalcopy % Uncomment for camera-ready version, but NOT for submission.

%\preprintcopy % Uncomment for the preprint version, but NOT for submission.

\begin{document}


\maketitle

\begin{abstract}
    Trajectory-User Linking (TUL), which aims to associate unlabeled spatial trajectories to a user or entity that generated them is a common task in transportation and mobility applications.
    Often times, data in such applications is generated as part of a data stream and may be affected by distributional shifts.
    For this reason, training TUL models \textit{online} by incrementally adapting them to new data can be beneficial in many cases.
    Nevertheless, online TUL has not been studied by existing research.
    To bridge this gap, we perform comprehensive online learning evaluations of some the most successful TUL techniques.
    We further propose a novel embedding approach for online TUL, which we call \underline{M}ult\underline{i}-\underline{L}evel \underline{E}mbedding \underline{S}haring (MiLES).
    MiLES involves partially sharing embeddings for locations within neighborhoods of multiple size levels, allowing the shared embeddings to receive more frequent updates and therefore enabling faster adaptation to new trajectory data.
    Our evaluations on multiple real-world datasets show that MiLES significantly improves the performance of existing state-of-the-art TUL approaches in an online learning setting.
\end{abstract}

\section{Introduction}\label{sec:intro}

Thanks to the proliferation of smartphones, wearables and other devices supporting location services, mobility data has become abundant, spawning a variety of machine learning tasks in the process \citep{rehmanMiningPersonalData2015,zhengTrajectoryDataMining2015}.
One such task is Trajectory User Linking (TUL), introduced by \citet{gaoIdentifyingHumanMobility2017}.
- Trajectory: temporally ordered sequence of so-called check-ins at visited locations or points of interest (POIs)
- TUL involves linking trajectories, to the users or entities that created them \citep{gaoIdentifyingHumanMobility2017}
- TUL has practical applications in disease control, law enforcement, ride-sharing, location recommendation and many more \citep{haoUnderstandingUrbanPandemic2020,gaoIdentifyingHumanMobility2017}
- Previous works on TUL achieved remarkable results using approaches based on recurrent neural networks and, more recently, on transformer architectures.
- These works are limited to conventional batch-learning where all training data is available at once.
- However, when establishing a new location-based service for instance, the amount of initial data available for training may be insufficient for achieving optimal results with this approach, due to the fact that in many real-world applications, data is instead generated one sample or chunk at a time in the form of a \textit{data stream}.
- Furthermore, such data streams are commonly affected by temporal dependencies and shifts in their distributions in the form of concept drift \citep{gomesSurveyEnsembleLearning2017}.
- For mobility data, distributional shifts notably occurred as a result of the COVID-19 pandemic \citep{borkowskiLockdownedEverydayMobility2021} but may, for example, also occur due to seasonality, new points of interest, or changes in user activity.
- Therefore, incremental adaptation to new trajectories through the use of \textit{online learning} can be essential in real-world applications of TUL.

According to \citet{bifetMOAMassiveOnline2010}, an online machine learning model operating in a streaming environment must be able to
% Remove this? Or does it help understand what online learning means?
\begin{enumerate}
    \item[R1:] process a single instance at a time,\label{rq:single_instance}
    \item[R2:] process each instance in a limited amount of time,\label{rq:limited_time}
    \item[R3:] use a limited amount of memory,\label{rq:limited_memory}
    \item[R4:] predict at any time,\label{rq:predict_any_time}
    \item[R5:] adapt to changes in the data distribution.\label{rq:adapt_to_drift}
\end{enumerate}

% While most existing TUL approaches can easily be adapted to fulfill these requirements by executing a training step every time a new trajectory is received, the requirements to predict at any time (R4) and adapt to changes in the data distribution (R5)
- In theory, most existing TUL approaches can easily be adapted to fulfill these requirements by executing a training step every time a new trajectory is received
- Approach for embedding check-ins of existing TUL approaches may negatively impact their ability to predict at any time (R4) and adapt to changes in the data distribution (R5)
- Most models embed check-ins using a lookup table containing embedding vectors for each individual POI
- Due to large number of different locations, visits at most individual POIs is generally infrequent \citep{chenMutualDistillationLearning2022a}
- Only embeddings of visited POIs have non-zero gradient
- Not a big issue in batch-learning, due to training on dataset collected over a long timespan
- Causes embeddings to adapt slowly in online TUL
- Sharing embeddings between multiple locations causes them to have significantly lower gradient sparsity
- Intuitively knowledge about locations is shared enabling the network to interpret ones with that received few check-ins
- Sharing embeddings based on spatial proximity/neighborhoods *makes sense*
- POI's withing the same neighborhood often share similar characteristics, like function (i.e. working, living, recreation) or affluence
- Using proximity as an inductive bias also requires no additional information
- Embedding gradients get less sparse with increasing neighborhood size
- However, there is a trade-off between density and informational content as embeddings also become less specific with increasing neighborhood-size
- Therefore, sharing between large neighborhoods may be helpful at the start of training and after concept drifts but inhibits the models capability to accurately interpret precise locations
- For this reason, we propose multi-level embedding sharing (MiLES), which pools different parts of the embeddings at different levels of neighborhood sizes
- MiLES divides embeddings into multiple segments, where each segment is shared between neighborhoods of varying sizes to enable a spectrum of features with varying informational content and adaptation speed
- Different segments may be prioritized based on whether density/adaptation speed or feature informational content is more important at the respective stage of the online learning process

- To analyse the effectiveness of MiLES, we first perform a broad evaluation of existing (state-of-the-art) TUL techniques in an online learning setting (i)
- We then repeat the same evaluations with each of the techniques adapted to use our proposed MiLES embedding technique finding significant performance improvements (ii)
- Lastly, we perform an ablation study for each component of MiLES and conduct an in-depth analysis of its functionality (ii)


\section{Preliminaries}\label{sec:preliminaries}

In the following we will introduce the basic concepts underlying online trajectory user linking:

% As previously noted, the fundamental units of information for TUL are so-called check-ins which are defined as follows:

\begin{definition}[Check-In]
    A check-in is a tuple $\vc = (u, t, \vl)$ containing a user identifier $u$ from a set of participating users $\sU$, timestamps $t$ and geo-coordinates $\vl$ of the visited location.
    Check-ins often additionally contain a unique identifier $p\in\sP$, where $\sP$ is a finite set of POI identifiers.
\end{definition}

\begin{definition}[Trajectory]
    A trajectory is a chronologically ordered sequence of $n$ check-ins $[(u, t_0, \vl_0), ..., (u, t_n, \vl_n)]$ generated by a user $u$ within a specific timeframe $\tau$. If the check-ins of a trajectory lack a user label $u$, they are called \textit{unlinked}.
\end{definition}

\begin{definition}[Trajectory User Linking] %TODO: convert to problem instead of definition
    The task of trajectory user linking is to learn a function $f$ on a training set of historical trajectories $\sT_{(\rm{train})}$, that links a set of \textit{unlinked} trajectories $\sT^{(\rm{test})} = \{T_0, ..., T_m\}$ to the users $\sU^{(\rm{test})} = \{u_0,...,u_m\}$ that generated them. Formally, the objective of TUL can be described as
    \begin{equation}
        \min_{\bm{\theta}} \sum_{i=0}^{m} L(f(T_i; \bm{\theta}, \sT^{(\rm{train})}), u_i), \ T_i \in \sT^{(\rm{test})}, u_i \in \sU^{(\rm{test})},
    \end{equation}
    where $L$ is a loss function that quantifies the predictive error and $\bm{\theta}$ are the model parameters to be optimized.
\end{definition}

% TODO: move this to approach?
Due to requirements \textbf{R1-R5} the above definition of TUL, where the model is trained and tested on seperate, self-contained is impractical for an online learning setting.
Instead, online TUL can more accurately be analyzed using the \textit{prequential} or \textit{interleaved test-then-train} evaluation scheme \citep{bifetMOAMassiveOnline2010}, where each new sample is first used to validate and then to train the underlying model.
According to this scheme, the optimization goal of online TUL can be formulated as
\begin{equation} % Notation kinda ugly 
    \min_{\bm{\theta}_0, ..., \bm{\theta}_{m-1}} \sum_{i=1}^{m-1} L(f(T_{i}; \bm{\theta}_{i-1}, \sT^{(\rm{stream})}_{:i-1}), u_{i}), \ T_i \in \sT^{(\rm{stream})}, u_i \in \sU^{(\rm{stream})},
\end{equation}
where $\bm{\theta}_{i-1}$ denotes the model parameters at timestep $i-1$ and $\sT^{(\rm{stream})}_{:i-1}$ is the set of all samples in the data stream up until timestep $i-1$.
A key difference between the batch-mode and the online learning formulation of TUL is that for the latter, the parameters at each step of the training process contribute equally to the performance of the model, whereas for the former, only the parameters achieved at the very end of the training process are relevant.
For this reason, the sparsity issue of existing location embedding techniques mentioned in \ref{sec:intro} can have a negative impact in online TUL.


\section{Related Work}

While trajectory user linking itself is a relatively recent task, several older approaches from adjacent fields are also suitable for it.
For instance, using the longest common subsequence algorithm \citep{yingSemanticTrajectoryMining2011} one can predict the user label of an unlinked trajectory based on the trajectory sharing the longest sub-trajectory with the input.

By encoding trajectories solely based on the number of occurrences of each POI, using a bag-of-words approach \citep{mikolovEfficientEstimationWord2013}, conventional classification techniques such as linear discriminant analysis or support vector machines can be used for TUL.

More recent approaches starting with TUL via Embedding and RNN (TULER) \citep{gaoIdentifyingHumanMobility2017}, instead employ an embedding scheme that maintains the order of check-ins by replacing each location in a trajectory with its corresponding real-valued vector stored in an embedding table $Z \in \sR^{|\sP| \times D}$, where $|P|$ is the number of unique locations and $D$ the embedding dimension.
\citet{gaoIdentifyingHumanMobility2017} introduced three different TULER variants, TULER-G, TULER-L and BiTULER, combining this embedding approach with either a GRU- \citep{choPropertiesNeuralMachine2014a}, an LSTM- or a bidirectional LSTM network \citep{hochreiterLongShort1997}.
In subsequent works, various extensions of TULER were proposed.
TULVAE \citep{zhouTrajectoryUserLinkingVariational2018} for instance, combines an LSTM classifier with a variational autoencoder \citep{kingmaAutoEncodingVariationalBayes2022a}, which receives classifications as additional input.
With this approach, the LSTM classifier is additionally trained to aid the reconstruction of trajectories.
DeepTUL proposed by \citet{miaoTrajectoryUserLinkingAttentive2020} instead adds a historical attention module that generates a context vector based on the user IDs of all previous check-ins that share the same locations and time-slots as the input trajectory.
This context vector is then provided as an additional input to an RNN classifier.

With their advancement in other machine learning disciplines, newer studies on TUL have increasingly focused on transformer-based approaches.
The T3S model \citep{yangT3SEffectiveRepresentation2021}, for example, combines a transformer- and an LSTM-encoder for encoding trajectories before feeding them to a classification layer.
Unlike most other TUL approaches, T3S embeds check-ins by mapping them to cells of a square grid and using the respective grid cells as the keys for a lookup table.
Consequently, locations within the same grid cell share their embeddings and no POI information is removed.
uses a transformer that encodes the sequence of visited grid cells which is concatenated with an LSTM embedding of the coordinate sequence

In a similar fashion, the purely transformer-based TULHOR \citep{alsaeedTrajectoryUserLinkingUsing2023a} embeds check-ins based on their position within a hexagonal grid.
The algorithm supplements the grid-based embeddings with conventional POI embeddings as well as with the embeddings of the grid cells visited along the fastest routes between individual check-in locations.
Seeking to combine the benefits of transformers and RNNs, \citet{chenMutualDistillationLearning2022a} proposed MainTUL, which simultaneously trains one of both types of models on trajectories augmented with previous trajectories of the same user.
To enhance the models performance, the outputs of the transformer are distilled into the RNN.

Further TUL models include GNNTUL \citep{zhouTrajectoryUserLinkingGraph2021a} and AttnTUL \citep{chenTrajectoryUserLinkingHierarchical2024}, which process trajectories using graph neural networks, as well as the Siamese neural network TULSN \citep{yuTULSNSiameseNetwork2020a}.



\section{Approach}\label{sec:approach}

As previously described, sharing embeddings between a greater number of locations increases their gradient density but reduces their informational content.
% Add proof for this? Or is this trivial? 
Embedding techniques of existing TUL approaches generally prioritize informational content by, if at all, sharing embeddings within small neighborhoods, due to the fact that the negative effect of gradient sparsity can be mitigated by training on a large amount of data.

However, this is not possible in the online learning setting.
Instead, due to the necessity to quickly adapt to new data, both density and informational content may vary in terms of impact on the model's performance throughout a data stream.
By proposing multi-level embedding sharing (MiLES) we therefore seek to provide an alternative solution by generating embedding features that cover a wide area of the density-information spectrum.
In the following, we will give an in-depth description of the functionality of MiLES, which is also depicted in Figure~\ref{fig:approach}.

% As described in Section~\ref{sec:intro}, the POI-based embedding approach, of most existing TUL models may limit their ability to quickly adapt to a data stream.
% To mitigate this issue while maintaining the model's ability to distinguish between precise locations, we propose the multi-level embedding sharing approach (MiLES).

Like TULHOR's embedding approach \citep{alsaeedTrajectoryUserLinkingUsing2023a}, MiLES maps locations to a grid created as a tiling of regular hexagons, which we selected due to their better representation of euclidean distance based neighborhoods.
However, unlike in TULHOR, we repeat this process multiple times with increasing cell sizes and therefore increasing levels of aggregation.

\begin{figure}[ht]
    \centering
    \resizebox{\textwidth}{!}{
        % Declare axial_round function
\pgfmathdeclarefunction{axial_round_x}{2}{%
    \pgfmathparse{round(#1)} \let\xgrid\pgfmathresult
    \pgfmathparse{round(#2)} \let\ygrid\pgfmathresult
    \pgfmathparse{#1 - \xgrid} \let\xrem\pgfmathresult
    \pgfmathparse{#2 - \ygrid} \let\yrem\pgfmathresult
    \pgfmathparse{round(\xrem + 0.5 * \yrem) * (\xrem * \xrem >= \yrem * \yrem)} \let\dx\pgfmathresult
    \pgfmathparse{\xgrid + \dx}%
}

\pgfmathdeclarefunction{axial_round_y}{2}{%
    \pgfmathparse{round(#1)} \let\xgrid\pgfmathresult
    \pgfmathparse{round(#2)} \let\ygrid\pgfmathresult
    \pgfmathparse{#1 - \xgrid} \let\xrem\pgfmathresult
    \pgfmathparse{#2 - \ygrid} \let\yrem\pgfmathresult
    \pgfmathparse{round(\yrem + 0.5 * \xrem) * (\xrem * \xrem < \yrem * \yrem)} \let\dy\pgfmathresult
    \pgfmathparse{\ygrid + \dy}%
}

% Declare pixel_to_flat_hex function
\pgfmathdeclarefunction{pixel_to_flat_hex_x}{3}{%
    \pgfmathparse{(2 / 3 * (#1 - #3)) / #3} \let\q\pgfmathresult
    \pgfmathparse{(-1 / 3 * (#1 - #3) + sqrt(3) / 3 * (#2- sqrt(3) / 2 * #3)) / #3} \let\r\pgfmathresult
    \pgfmathparse{axial_round_x(\q, \r)}%
}

\pgfmathdeclarefunction{pixel_to_flat_hex_y}{3}{%
    \pgfmathparse{(2 / 3 * (#1 - #3)) / #3} \let\q\pgfmathresult
    \pgfmathparse{(-1 / 3 * (#1 - #3) + sqrt(3) / 3 * (#2- sqrt(3) / 2 * #3)) / #3} \let\r\pgfmathresult
    \pgfmathparse{axial_round_y(\q, \r)}%
}

\pgfmathdeclarefunction{axial_to_oddq_row}{2}{%
    \pgfmathparse{#2 + (#1 - mod(#1, 2)) / 2} % Row calculation
}

\tdplotsetmaincoords{70}{20} % Set viewing angle

\begin{tikzpicture}[tdplot_main_coords, cross/.style={path picture={
                        \draw[black]
                        (path picture bounding box.south east) -- (path picture bounding box.north west) (path picture bounding box.south west) -- (path picture bounding box.north east);
                    }}]

    % Viridis colormap as RGB triples
    \def\viridisColors{{"0.267004 0.004874 0.329415", "0.127568 0.566949 0.550556",
                "0.369214 0.788888 0.382914", "0.993248 0.906157 0.143936"}}

    % Define check-in locations as separate x and y coordinate lists
    \def\x{8.1}
    \def\y{5}

    % Define table parameters
    \newcommand{\tableWidth}{2}
    \newcommand{\tableSpacing}{1.5}
    \newcommand{\nColumns}{4}

    % Grid and image configuration
    \def\hexsize{1} % Radius of a single hexagon
    \def\xsteps{6}   % Number of hexagons in the x direction
    \def\ysteps{6}   % Number of hexagons in the y direction
    \pgfmathsetmacro{\gridWidth}{\xsteps * \hexsize * 1.5 + 2 * \hexsize}
    \pgfmathsetmacro{\gridHeight}{\ysteps * \hexsize * sqrt(3) + 1.5 * \hexsize * sqrt(3) - 0.04}

    % First node with the map scaled to match the grid dimensions

    \node[canvas is xy plane at z=0, draw, fill=white, anchor=south west] at (0, 0)
    {\includegraphics[width=\gridWidth cm, height=\gridHeight cm]{figs/map.png}};

    \coordinate (catEmbedding) at ({\gridWidth + 3 * \tableSpacing + \tableWidth}, {\gridHeight * 17 / 18 / 2}, 0);

    % Add Level 0 label on the xz plane below catEmbedding
    \begin{scope}[canvas is xz plane at y = {\gridHeight * 17 / 18 / 2}]
        \node [transform shape, anchor=north west] at (catEmbedding |- {{0, 0}}) {\Large Level 0};
    \end{scope}

    \begin{scope}[canvas is xy plane at z=0]
        \pgfmathsetmacro{\selRow}{9}
        \pgfmathsetmacro{\nrows}{18}
        \pgfmathsetmacro{\rowHeight}{\gridHeight / 18}
        \pgfmathsetmacro{\tableHeight}{\rowHeight * \nrows}
        \pgfmathsetmacro{\columnWidth}{\tableWidth / (\nColumns+1)}
        \coordinate (tbase) at ({\gridWidth + \tableSpacing}, {(\gridHeight - \tableHeight) / 2});


        \pgfmathparse{\viridisColors[0]};
        \definecolor{currentColor}{rgb}{\pgfmathresult};
        % Background rectangle for table - now centered
        \filldraw[draw=black, fill=currentColor, fill opacity=0.75]
        (tbase) rectangle ++ (\tableWidth, \tableHeight);

        % Draw vertical grid lines
        \foreach \i in {1,...,\nColumns} {
                \draw[white] (tbase) ++ ({\i*\columnWidth}, 0)
                -- ++ (0, \tableHeight);
            }

        % Draw grid lines
        \foreach \i in {0,...,\nrows} {
                \draw[black] (tbase) ++ (0, {\i*\rowHeight})
                -- ++ (\tableWidth, 0);
            }

        % Highlight selected row in embedding table
        \pgfmathsetmacro{\selRowY}{\selRow * \rowHeight}
        \coordinate (selRowLeft) at ($ (tbase) + (0, {\selRowY + 0.5*\rowHeight}) $);
        \draw[red, thick]
        (tbase) ++ (0, \selRowY) rectangle ++ (\tableWidth, \rowHeight);

        % Mark check-in locations on the top plane
        \draw[-Latex, red, thick] (\x, \y) -- (selRowLeft);

        % Draw box around catEmbedding
        \draw[black] (catEmbedding) rectangle ++ ({4*\tableWidth}, \rowHeight);

        % Draw blue catEmbedding part
        \fill[currentColor, opacity=0.75] (catEmbedding) rectangle ++ (\tableWidth, \rowHeight);


        % Draw arrow to catEmbedding with intermediate circle node
        \coordinate (circlePos) at ($ (selRowLeft) + (\tableWidth + 1.25*\tableSpacing, 0) $);
        \node [draw,circle,cross, minimum width=12 pt] (crossNode) at (circlePos) {};

        % Split arrow into two segments
        \draw[-Latex, red, thick]
        (selRowLeft) ++ (\tableWidth, 0) --
        (crossNode.west);
        \draw[-Latex, red, thick]
        (crossNode.west) ++ (13 pt, 0) --
        ($ (catEmbedding) + (0, {\rowHeight*0.5}) $);

        % Add w_0 label pointing to circle
        \node [above=.5*\tableSpacing] (w0) at (crossNode) {\LARGE $w_0$};
        \draw[-Latex, thick, red] (w0) -- (crossNode);

        % Draw catEmbedding vertical lines
        \foreach \i in {1,...,\nColumns} {
                \draw[white] (catEmbedding) ++ ({\i*\columnWidth}, 0)
                -- ++ (0, \tableHeight);
            }

    \end{scope}



    % Loop to draw multiple hexagonal grids
    \foreach \z/\steps/\selRow [count=\n from 1] in {3/18/7,6/12/5,9/6/2} {


            % Mark check-in locations at different z-levels
            \pgfmathsetmacro{\zminus}{\z - 3}
            % Draw arrows
            \draw[-, color=red] (\x,\y,\zminus) -- ++ (0, 0, 3);
            \fill[red] (\x,\y,\zminus) circle (.125);

            % Select the color
            \pgfmathparse{\viridisColors[\n]};
            \definecolor{currentColor}{rgb}{\pgfmathresult};

            % Draw concatenated embedding
            % \fill

            \pgfmathsetmacro{\hexsize}{\gridWidth / (1.5 * \steps + 2)}

            \begin{scope}[canvas is xy plane at z=\z]
                % Draw rectangle (background for the grid)
                \filldraw[draw=black, fill=currentColor, fill opacity=0.75] (0, 0) rectangle ++ (\gridWidth, \gridHeight);

                % Loop to create hexagonal grid
                \foreach \x in {0,...,\steps} {
                        \foreach \y in {0,...,\steps} {
                                % Offset every second row to create a hexagonal grid
                                \pgfmathsetmacro{\xoffset}{(\x + 1) * \hexsize * 1.5}
                                \pgfmathsetmacro{\yoffset}{\y * \hexsize * sqrt(3) + mod(\x, 2) * \hexsize * sqrt(3) / 2}

                                % Draw hexagon
                                \draw[darkgray]
                                (\xoffset, \yoffset) -- ++(60:\hexsize) -- ++(120:\hexsize)
                                -- ++(180:\hexsize) -- ++(240:\hexsize) -- ++(300:\hexsize) -- cycle;
                                % Conditional to draw extra line only for first hexagon in last row
                                \ifnum\y=\steps
                                    \ifnum\x=0 
                                        \draw[darkgray]
                                        (\xoffset, \yoffset) ++ (-\hexsize, {sqrt(3)*\hexsize}) -- ++(300:-\hexsize);
                                    \fi
                                    \ifnum\x=\steps 
                                        \draw[darkgray]
                                        (\xoffset, \yoffset) ++ (0, {sqrt(3)*\hexsize}) -- ++(60:\hexsize);
                                    \fi
                                \fi
                            }
                    }



                % Draw embedding table

                % Calculate consistent row height based on maximum steps (18)
                \pgfmathsetmacro{\nrows}{round(\steps*0.8)}
                \pgfmathsetmacro{\rowHeight}{\gridHeight / 18}
                \pgfmathsetmacro{\tableHeight}{\rowHeight * \nrows}
                \pgfmathsetmacro{\columnWidth}{\tableWidth / (\nColumns+1)}

                \coordinate (tbase) at ({\gridWidth + \tableSpacing}, {(\gridHeight - \tableHeight) / 2});

                % Background rectangle for table - now centered
                \filldraw[draw=black, fill=currentColor, fill opacity=0.75]
                (tbase) rectangle ++ (\tableWidth, \tableHeight);

                % Draw vertical grid lines
                \foreach \i in {1,...,\nColumns} {
                        \draw[white] (tbase) ++ ({\i*\columnWidth}, 0)
                        -- ++ (0, \tableHeight);
                    }

                % Draw grid lines
                \foreach \i in {0,...,\nrows} {
                        \draw[black] (tbase) ++ (0, {\i*\rowHeight})
                        -- ++ (\tableWidth, 0);
                    }

                % Highlight selected row in embedding table
                \pgfmathsetmacro{\selRowY}{\selRow * \rowHeight}
                \coordinate (selRowLeft) at ($ (tbase) + (0, {\selRowY + 0.5*\rowHeight}) $);
                \draw[red, thick]
                (tbase) ++ (0, \selRowY) rectangle ++ (\tableWidth, \rowHeight);

                % Highlight visited hexagons
                % Compute hex coordinates
                \pgfmathparse{pixel_to_flat_hex_x(\x, \y, \hexsize)} \let\hexcoordx\pgfmathresult
                \pgfmathparse{pixel_to_flat_hex_y(\x, \y, \hexsize)} \let\hexcoordy\pgfmathresult

                \pgfmathparse{axial_to_oddq_row(\hexcoordx, \hexcoordy)} \let\hexcoordy\pgfmathresult

                \pgfmathsetmacro{\xoffset}{(\hexcoordx + 1) * \hexsize * 1.5}
                \pgfmathsetmacro{\yoffset}{\hexcoordy * \hexsize * sqrt(3) + mod(\hexcoordx, 2) * \hexsize * sqrt(3) / 2}

                % Draw hexagon
                \draw[red, thick]
                (\xoffset, \yoffset) -- ++(60:\hexsize) -- ++(120:\hexsize)
                -- ++(180:\hexsize) -- ++(240:\hexsize) -- ++(300:\hexsize) -- cycle;

                % Add red arrow from first hexagon only
                % Draw arrow to middle of right edge of highlighted row

                \draw[-Latex, red, thick]
                (\xoffset, \yoffset) ++ (60:\hexsize)
                -- (selRowLeft);

                % % Draw arrow to catEmbedding
                \coordinate (currentCatEmbedding) at ($ (catEmbedding) + ({\tableWidth*\n},0)$);
                % \draw[-Latex, red, thick] (selRowLeft) ++ (\tableWidth, 0) -- ($ (currentCatEmbedding) + (0, {\rowHeight*0.5}) $);

                % Draw arrow to catEmbedding with intermediate circle node
                \coordinate (circlePos) at ($ (selRowLeft) + (\tableWidth + 1.25*\tableSpacing, 0) $);
                \node [draw,circle,cross, minimum width=12 pt] (crossNode) at (circlePos) {};

                % Split arrow into two segments
                \draw[-Latex, red, thick]
                (selRowLeft) ++ (\tableWidth, 0) --
                (crossNode.west);
                \draw[-Latex, red, thick]
                (crossNode.south east) --
                ($ (currentCatEmbedding) + (0, {\rowHeight*0.5}) $);

                % Add w_0 label pointing to circle
                \node [above=.5*\tableSpacing] (w0) at (crossNode) {\LARGE $w_{\pgfmathprint{\n}}$};
                \draw[-Latex, thick, red] (w0) -- (crossNode);

                % Draw catEmbedding part
                \fill[currentColor, opacity=0.75] (currentCatEmbedding) rectangle ++ (\tableWidth, \rowHeight);

                % Draw catEmbedding vertical lines
                \foreach \i in {1,...,\nColumns} {
                        \draw[white] (currentCatEmbedding) ++ ({\i*\columnWidth}, 0)
                        -- ++ (0, \tableHeight);
                    }

            \end{scope}

            % Add Level 0 label on the xz plane below catEmbedding
            \begin{scope}[canvas is xz plane at y = {\gridHeight * 17 / 18 / 2}]
                \node [transform shape, anchor=north west] at (currentCatEmbedding |- {{0, 0}}) {\Large Level \pgfmathprint{\n}};
            \end{scope}
        }

    % Mark check-in locations on the top plane
    \fill[red] (\x,\y, 9) circle (.125);


\end{tikzpicture}



    }
    \caption{Visualization of the proposed multi-level embedding sharing (MiLES) technique with learnable weights $w_l$ and embedding dimensions $d_l$ for each level $l\in \{0,1,2,3\}$ and $\otimes$ representing scalar multiplication. Embedding dimensions are calculated according to Equation~\ref{eq:emb_dims} but are drawn equally sized for visual clarity here.}
    \label{fig:approach}
\end{figure}

Accordingly, we augment check-ins with an additional index $h_l$ for each embedding-level $l \in \{0, 1, ..., l^{(\rm{max})}\}$, which identifies the grid cell containing the check-in location.
If available, we leverage POI information by setting $h_0$ to the POI index.
We further initialize an embedding matrix $\mZ_{l}$ of shape $h_l^{\rm{(max)}} \times d_l$ for each level, where $d_l$ is the embedding dimension and $h_l^{\rm{(max)}}$ the maximum POI- or grid cell index.
To account for the decreasing informational content with increasing levels of aggregation, we assign smaller dimensions to higher-level embeddings.
We compute the individual embedding dimensions as
\begin{equation}\label{eq:emb_dims}
    d_l=\frac{d \cdot \alpha^{-l}}{\sum_{l=0}^{l^{\rm{(max)}}} \alpha^{-l}}, \ \alpha > 1
\end{equation}
where $d$ is the dimension of the final embedding and $\alpha$ is a hyperparameter.
By default we use $\alpha=2$ in our experiments.

To compute the final embedding, we concatenate all level-specific embeddings, each weighted by a learnable parameter $w_l$.
Using $\concat$ to represent vector concatenation, the embedding function $g$ can be defined as
\begin{equation}
    g(\vh; \mZ_0, \mZ_1, ..., \mZ_{l^{\rm{(max)}}} \vw) = \concat_{l=0}^{l^{\rm{(max)}}} \mZ_{l,h_l,:} \cdot w_l,
\end{equation}
where $\mZ_{l,h_l,:}$ is the row vector of $\mZ_l$ located at index $h_l$.

We select concatenation instead of summation to aggregate the level-specific embeddings to avoid interference between levels.
For a fixed total embedding dimension, this approach also significantly reduces the number of parameters compared to a purely POI-based or summed multi-level embedding, since sharing embeddings between multiple locations decreases the size of the higher level embedding matrices.
Because MiLES only requires an $O(1)$ table lookup and an $O(d_l)$ multiplication for each of its levels, it is also efficient in terms of computational complexity.

By multiplying the level-specific embeddings $\mZ_{l,h_l,:}$ with learnable parameters $w_l$, we allow the average embedding-norm to be optimized on a per-level basis throughout the online learning process.
This can be interpreted as an attention mechanism that adapts the model's attention to individual embedding features based on previously observed data.
With this mechanism, the model can prioritize different embedding features based on the importance of gradient density or informational content for the current state of the data stream.

For a given embedding level $l$, the gradient of the final MiLES embedding $\vg_l$ with respect to the selected embedding vector $\mZ_{h_l, l,:}$ is given by
$$\frac{\delta \vg_l}{\delta \mZ_{h_l, l,:}} = \frac{\delta w_l * \mZ_{h_l, l,:}}{\delta \mZ_{h_l, l,:}} = w_l.$$
This means that the learnable weighting $w_l$ not only scales the embeddings but also influences the magnitude of their gradients.
As a result, it effectively acts as a level-specific learning rate, allowing for adaptive gradient scaling across different embedding levels.

In order not to affect the initial probability distribution of the embedding vectors, we use a starting value of 1 for all $w_l$ in our experiments.

% Remove this?
Theoretically, the effect of the weighting mechanism described above could also simply be achieved by the learning process adapting the respective weights of the input layer $\theta_l^{(\rm{in})}$ to $w_l\theta_l^{(\rm{in})}$. However, we hypothesize that adding dedicated embedding attention weights can nevertheless be beneficial, since it allows for independent adjustments of the scale and orientation of the individual embeddings.
% Add pseudo-code?

\section{Experiments}

To evaluate the impact of the proposed MiLES approach and its individual components on the performance of existing TUL models in a data stream setting, we perform a multitude of experiments.

We conduct our experiments using the popular public trajectory datasets GeoLife, Foursquare-NYC and Foursquare-TKY, which we pre-process according to \citet{chenMutualDistillationLearning2022a} by splitting each trajectory into shorter trajectories with a maximum length of 24 hours for Foursquare-NYC and Foursquare-TKY and 3 hours for GeoLife and select the most active users.
For more information on the selected datasets, refer to Table~\ref{tab:datasets}.

To adhere to the test-then-train evaluation scheme described in Section~\ref{sec:preliminaries}, we process the datasets one trajectory at a time by using each of them to first evaluate and subsequently perform a training step with the respective TUL model.
For the latter we employ the Adam optimization method \cite{kingmaAdamMethodStochastic2017a}.
In terms of data modalities, we use GPS coordinates, timestamps and, for the Foursquare datasets, which include such information, POI identifiers.
We further provide hour-specific lookup embeddings as additional model inputs, as was commonly done in previous work \citep[see e.g.][]{chenMutualDistillationLearning2022a,miaoTrajectoryUserLinkingAttentive2020}.
For better comparability, we omit any model components that require additional information like the mobility flows used by TULHOR.
For the evaluation of MainTUL and DeepTUL, which require historical data as inputs, we maintain a buffer of the most recent 1000 trajectories during each run.

For approach-specific hyperparameters, we use the authors suggested default values and tune other hyperparameters like the learning rate, number of layers and model size on the first 5000 trajectories of Foursquare-TKY.
We use the same approach to tune the hyperparameters of the proposed embedding technique, which we implement by dividing the map-area containing all check-in locations based on multiple hexagonal grids, with varying numbers of rows.
According to the best-performing configuration in our tuning runs, we use a total of 3 embedding sharing levels with 200 grid rows for the base level and halve the number of rows for each subsequent embedding level.
In case of the GeoLife dataset, which lacks POI information, we use an additional grid-based embedding level with 800 grid-rows to replace the POI embedding level.

Unless stated otherwise, we repeat each of our experiments with 5 different random seeds and report the averages over all runs.
We implement each of the evaluated TUL approaches using PyTorch \cite{paszkePyTorchImperativeStyle2019a}.

\begin{table}[hb]
    \centering
    \caption{Datasets used for experimental evaluation.}
    \label{tab:datasets}
    \begin{tabular}{@{}lrrrrr@{}}
        \toprule
        Dataset                                          & \# Users & \# Trajectories & \# Check-Ins & \# POIs & Timespan  \\ \midrule
        \multirow{2}{*}{Foursquare-NYC} \footnotemark[1] & 800      & 61,218          & 196,435      & 34,383  & 10 months \\
                                                         & 400      & 35,510          & 137,886      & 25,443  & 10 months \\ \midrule
        \multirow{2}{*}{Foursquare-TKY} \footnotemark[1] & 800      & 70,007          & 324,564      & 38,212  & 10 months \\
                                                         & 400      & 44,955          & 248771       & 28,286  & 10 months \\ \midrule
        \multirow{2}{*}{GeoLife}  \footnotemark[2]       & 150      & 25,611          & 1,284,208    & 0       & 64 months \\
                                                         & 75       & 23,290          & 1,187,510    & 0       & 64 months \\ \bottomrule
    \end{tabular}
\end{table}

\footnotetext[1]{Publicly available at \url{https://sites.google.com/site/yangdingqi/home/foursquare-dataset}}
\footnotetext[2]{Publicly available at \url{https://www.microsoft.com/en-us/download/details.aspx?id=52367}}

In the following, we will describe the results of our prequential evaluations.
To evaluate their suitability for online learning applications, we ran experiments with a variety of existing TUL approaches and their original location embedding techniques.
The results of these experiments for the dataset variants with the higher user counts are depicted in Table~\ref{tab:models}.
For an overview of the complete results, please refer to Appendix~\ref{sec:full_results}.

As Table~\ref{tab:models} shows, the bidirectional LSTM-based BiTULER achieved the overall best performance, with the only exception being the GeoLife dataset where DeepTUL yielded a higher top-1 accuracy and macro f1 score.
A likely reason for this is that as the simplest model with the lowest number of parameters allowing for faster adaptation to the distribution of the data stream.
The same effect may also explain the overall lower performance of the transformer-based MainTUL, T3S and TULHOR models compared to the RNN-based alternatives.

\begin{table}[hb]
    \caption{Top-1 accuracy (Acc@1), top-5 accuracy (Acc@5) and macro F1 score [\%] of TUL models averaged over prequential evaluation runs on the Foursquare and Geolife datasets with 800 and 150 users respectively.}
    \label{tab:models}
    \begin{tabular}{lccccccccc}
        \toprule
        Dataset  & \multicolumn{3}{c}{Foursquare-NYC} & \multicolumn{3}{c}{Foursquare-TKY} & \multicolumn{3}{c}{GeoLife}                                                                                                             \\
        \cmidrule(lr){2-4} \cmidrule(lr){5-7} \cmidrule(lr){8-10}
        Model    & Acc@1                              & Acc@5                              & Macro F1                    & Acc@1           & Acc@5           & Macro F1        & Acc@1           & Acc@5           & Macro F1        \\
        \midrule
        Bi-TULER & \bfseries 60.12                    & \bfseries 67.20                    & \bfseries 57.83             & \bfseries 61.28 & \bfseries 73.08 & \bfseries 58.96 & \bfseries 37.56 & 70.85           & 26.69           \\
        TULVAE   & 59.79                              & 66.77                              & 57.32                       & 54.19           & 64.92           & 49.85           & 37.08           & 70.45           & 25.25           \\
        DeepTUL  & 58.72                              & 65.48                              & 56.60                       & 59.14           & 70.66           & 56.90           & 36.32           & \bfseries 72.64 & \bfseries 29.82 \\
        MainTUL  & 55.67                              & 62.61                              & 53.01                       & 56.81           & 69.18           & 54.09           & 34.00           & 70.26           & 21.76           \\
        T3S      & 52.98                              & 60.28                              & 49.50                       & 53.65           & 66.30           & 50.38           & 35.25           & 71.11           & 21.52           \\
        TULHOR   & 53.85                              & 61.13                              & 50.40                       & 54.39           & 67.24           & 51.06           & 34.65           & 72.46           & 24.92           \\
        \bottomrule
    \end{tabular}
\end{table}

We subsequently repeated our experiments with different TUL approaches, replacing their original embedding modules with the proposed MiLES technique.
To disentangle the effect of the embedding dimensionality, we configured MiLES to use the same final embedding dimension of 1024 that we previously used in the evaluation of the original TUL models.
Table~\ref{tab:miles_improvements} shows the relative change in performance metrics measured in percentage points achieved with this adaptation.
The results demonstrate that MiLES yielded significant improvements for all datasets in terms of top-1 accuracy, top-5 accuracy and macro F1 score.
For the Foursquare datasets with 800 users, the improvements are especially large in terms of top-5 accuracy, which may be due to the embeddings with a higher abstraction level disproportionately helping to identify the correct user group.
The largest improvements can be seen on the GeoLife trajectories, where MiLES increased the top-5 accuracy by up to 6.98 percentage points and the top-1 accuracy by up to 8.72 percentage points, a relative increase of 23.21\%.
MiLES likely causes larger performance gains for GeoLife due to its lack of POI information.
Since existing techniques only use single grid-based embedding in place of the more meaningful POI-based embedding \citep{yangT3SEffectiveRepresentation2021} to solve this issue, the additional embedding levels of MiLES may be especially valuable in this case.

The observed improvements are also consistent across all models including T3S and TULHOR, whose original embedding techniques already include a single level grid-based embedding, showing that MiLES' multi-level approach offers significant benefits over existing approaches in an online learning setting.


\begin{table}[ht]
    \caption{Change of top-1 accuracy (Acc@1), top-5 accuracy (Acc@5) and macro F1 score metrics in percentage points when using MiLES over each models' original embedding technique for the dataset variants with higher user-counts.}
    \label{tab:miles_improvements}
    \begin{tabular}{lccccccccc}
        \toprule
        Dataset  & \multicolumn{3}{c}{Foursquare-NYC}                    & \multicolumn{3}{c}{Foursquare-TKY}                    & \multicolumn{3}{c}{GeoLife}                                                                                                                                                                                                                                                                                                                                                                           \\
        \cmidrule(lr){2-4} \cmidrule(lr){5-7} \cmidrule(lr){8-10}
        Model    & Acc@1                                                 & Acc@5                                                 & Macro F1                                              & Acc@1                                                 & Acc@5                                                 & Macro F1                                              & Acc@1                                                 & Acc@5                                                 & Macro F1                                              \\
        \midrule
        Bi-TULER & {\cellcolor[HTML]{E8F59F}} \color[HTML]{000000} +1.49 & {\cellcolor[HTML]{BFE47A}} \color[HTML]{000000} +3.58 & {\cellcolor[HTML]{E3F399}} \color[HTML]{000000} +1.71 & {\cellcolor[HTML]{E8F59F}} \color[HTML]{000000} +1.44 & {\cellcolor[HTML]{CFEB85}} \color[HTML]{000000} +2.87 & {\cellcolor[HTML]{E8F59F}} \color[HTML]{000000} +1.44 & {\cellcolor[HTML]{36A657}} \color[HTML]{000000} +8.72 & {\cellcolor[HTML]{6BBF64}} \color[HTML]{000000} +6.98 & {\cellcolor[HTML]{7FC866}} \color[HTML]{000000} +6.26 \\
        TULVAE   & {\cellcolor[HTML]{E2F397}} \color[HTML]{000000} +1.79 & {\cellcolor[HTML]{BDE379}} \color[HTML]{000000} +3.73 & {\cellcolor[HTML]{DFF293}} \color[HTML]{000000} +2.04 & {\cellcolor[HTML]{CFEB85}} \color[HTML]{000000} +2.90 & {\cellcolor[HTML]{B1DE71}} \color[HTML]{000000} +4.29 & {\cellcolor[HTML]{C7E77F}} \color[HTML]{000000} +3.22 & {\cellcolor[HTML]{4BB05C}} \color[HTML]{000000} +8.01 & {\cellcolor[HTML]{75C465}} \color[HTML]{000000} +6.57 & {\cellcolor[HTML]{A7D96B}} \color[HTML]{000000} +4.74 \\
        DeepTUL  & {\cellcolor[HTML]{EEF8A8}} \color[HTML]{000000} +1.06 & {\cellcolor[HTML]{CBE982}} \color[HTML]{000000} +3.04 & {\cellcolor[HTML]{EBF7A3}} \color[HTML]{000000} +1.26 & {\cellcolor[HTML]{EFF8AA}} \color[HTML]{000000} +0.94 & {\cellcolor[HTML]{D9EF8B}} \color[HTML]{000000} +2.44 & {\cellcolor[HTML]{EFF8AA}} \color[HTML]{000000} +0.95 & {\cellcolor[HTML]{39A758}} \color[HTML]{000000} +8.59 & {\cellcolor[HTML]{7DC765}} \color[HTML]{000000} +6.31 & {\cellcolor[HTML]{87CB67}} \color[HTML]{000000} +5.97 \\
        MainTUL  & {\cellcolor[HTML]{E8F59F}} \color[HTML]{000000} +1.44 & {\cellcolor[HTML]{AFDD70}} \color[HTML]{000000} +4.33 & {\cellcolor[HTML]{E6F59D}} \color[HTML]{000000} +1.52 & {\cellcolor[HTML]{E5F49B}} \color[HTML]{000000} +1.64 & {\cellcolor[HTML]{C1E57B}} \color[HTML]{000000} +3.49 & {\cellcolor[HTML]{E6F59D}} \color[HTML]{000000} +1.56 & {\cellcolor[HTML]{45AD5B}} \color[HTML]{000000} +8.19 & {\cellcolor[HTML]{82C966}} \color[HTML]{000000} +6.18 & {\cellcolor[HTML]{93D168}} \color[HTML]{000000} +5.50 \\
        T3S      & {\cellcolor[HTML]{E3F399}} \color[HTML]{000000} +1.71 & {\cellcolor[HTML]{C9E881}} \color[HTML]{000000} +3.13 & {\cellcolor[HTML]{DDF191}} \color[HTML]{000000} +2.10 & {\cellcolor[HTML]{E3F399}} \color[HTML]{000000} +1.71 & {\cellcolor[HTML]{CFEB85}} \color[HTML]{000000} +2.84 & {\cellcolor[HTML]{E0F295}} \color[HTML]{000000} +1.90 & {\cellcolor[HTML]{6BBF64}} \color[HTML]{000000} +6.98 & {\cellcolor[HTML]{A5D86A}} \color[HTML]{000000} +4.80 & {\cellcolor[HTML]{9BD469}} \color[HTML]{000000} +5.18 \\
        TULHOR   & {\cellcolor[HTML]{E3F399}} \color[HTML]{000000} +1.71 & {\cellcolor[HTML]{C9E881}} \color[HTML]{000000} +3.16 & {\cellcolor[HTML]{DDF191}} \color[HTML]{000000} +2.11 & {\cellcolor[HTML]{E2F397}} \color[HTML]{000000} +1.86 & {\cellcolor[HTML]{D1EC86}} \color[HTML]{000000} +2.75 & {\cellcolor[HTML]{DDF191}} \color[HTML]{000000} +2.10 & {\cellcolor[HTML]{45AD5B}} \color[HTML]{000000} +8.20 & {\cellcolor[HTML]{A0D669}} \color[HTML]{000000} +5.03 & {\cellcolor[HTML]{8ECF67}} \color[HTML]{000000} +5.68 \\
        \bottomrule
    \end{tabular}
\end{table}
To evaluate the usefulness of the individual components of MiLES, we performed an ablation study, where we removed either individual embedding levels (-L1, -L2, -L3).
When removing one of the embedding levels, we excluded the affected level in our calculation of the individual embedding dimension (see Equation~\ref{eq:emb_dims}) to eliminate any potential changes introduced by a decrease in the total embedding size compared to the default MiLES configuration.
We further ablated the weighting of embedding levels through learnable parameters $w$ (-WL) and the level-dependent embedding dimensions (-VD).
In the case of the latter, we equally split the total embedding size $d$ between the levels instead of using Equation~\ref{eq:emb_dims} to calculate the individual sizes.
We ran these ablations on all previously described datasets, using BiTULER, which yielded the best performance in our previous evaluations, as the underlying model.
The results of the ablation study, displayed in Figure~\ref{fig:ablation}, reveal that embedding levels two and three contribute significantly to the average performance of TULER across all datasets.
Possibly due to a high correlation with the POI-based embeddings, level one provides only a slight increase in terms of macro F1 score compared to the embedding module without it.
However, since it requires only an additional lookup operation and may yield a larger benefit with a more comprehensive tuning of MiLES' hyperparameters, it is arguably still worthwhile to include embedding level one.
When removing the weighting of the individual levels (-WL), we find a decrease in all performance metrics, highlighting the effectivity of this adaptation.
For the model without varying embedding dimensions (-VD), the losses in terms of top-1 accuracy and macro F1 score are even larger, whereas the top-5 accuracy remains unchanged.
This effect may be caused by the smaller embedding size assigned to the more precise low-level embeddings, which reduces the model's ability to identify the exact user linked to a trajectory.
% Add PCA visualization of MiLES embeddings?
\begin{figure}[h]
    \centering
    \includegraphics[width=0.6\textwidth]{figs/ablation_all.pdf}
    \caption{Results of ablation study for the proposed embedding technique (MiLES), where we removed either one of the embedding levels (-L1, -L2, -L3), the weighting of embedding levels (-WL) or the varying dimensions of individual embedding levels. BiTULER was used as the classifier. Displayed values are averaged over all datasets.}
    \label{fig:ablation}
\end{figure}

\begin{figure}[h]
    \centering
    \includegraphics[width=\textwidth]{figs/discretization_rows_BiTULER.pdf}
    \caption{Performance on Foursquare-NYC relative to number of embedding levels and rows in the base level grid on the performance achieved with MiLES.}
    \label{fig:discretization}
\end{figure}

\begin{figure}[h]
    \centering
    \includegraphics{figs/emb_prio.pdf}
    \caption{Mean weighting factors $w_0, ..., w_3$ and number of unique users within the last 1000 trajectories for evaluation runs on Foursquare-NYC. Shaded areas represent the 1$\sigma$ range.}
    \label{fig:emb_prio}
\end{figure}

To enable investigating the dynamics of the learnable level-prioritization weights $w_l$, we recorded their values throughout multiple prequential evaluation runs on the 400 user variant of the Foursquare-NYC dataset.
Figure~\ref{fig:emb_prio} displays the value of the recorded weights.

On average, all level-specific weights increase throughout the training process.
Through this mechanism, the effective learning rate of the embedding parameters is increased, as explained in Section~\ref{sec:approach}.

In terms of differences between the individual weights, it can be seen that the weights of the embedding levels with lower gradient sparsity $w_2$ and $w_3$ initially increase slightly faster than the weight of the POI-based embedding $w_0$.
After 5000 trajectories have been observed, $w_0$ starts exceeding $w_1$ and $w_2$, indicating a benefit of more specific embeddings at this stage of the online learning process.
Coinciding with a drop






\section{Conclusion}

- proposed novel embedding approach for online TUL
- evaluated efficacy of approach in combination with a variety of TUL models on three real-world datasets
- found improvements for all combinations of models and datasets
- improvements in terms of top 5 accuracy were greatest
- challenging GeoLife dataset benefitted the most
- embedding technique can be used for any task that takes localization data as input
- in terms of models, RNN-based ones generally outperformed transformer based approaches
- likely due to their faster convergence thanks to fewer parameters

\bibliography{collas2025_conference}
\bibliographystyle{collas2025_conference}

\appendix
\section{Full Results}\label{sec:full_results}


\end{document}
